%=============================================================================================
%=============================================================================================
\section{M11 Physical Layout - Seepage Management Features}
\label{sec:seepagefeatures}
%=============================================================================================
%=============================================================================================


The extensive seepage management system is described in this chapter, and is divided into 8 areas as shown in Figure~\ref{fig:M11_seepage_areas}.

\begin{figure}[!h]
  \begin{center}
  \includegraphics[width=3.0in]{M11_seepage_areas.png}
  \caption[MIKE11 Seepage Management Areas]{MIKE11 Seepage Management Areas}
  \label{fig:M11_seepage_areas}
  \end{center}
\end{figure}

\clearpage

Detention areas and the associated canals, pumps, weirs, and culverts included in the model and constructed under the Central and Southern Florida Project and Comprehensive Everglades Restoration Plan are outlined in the following sections. They extend from the 8.5 SMA Detention Cell southward to the Frog Pond Detention Area (Figure~\ref{fig:c111mods}). Also included in this section is the L-31N Seepage Management Barrier.

In general, the detention areas are implemented in the model as a broad, shallow canals.  Canal cross sections define the boundaries of the detention areas. The implementation within M11 allows a more accurate representation of the detention area geometry and an accurate calculation of detained volumes than using overland flow cells.

\begin{figure}[!h]
  \begin{center}
  \includegraphics[width=4.0in]{c111mods.png}
  \caption{MWD and C-111 Area map}
  \label{fig:c111mods}
  \end{center}
\end{figure}


\clearpage
%=============================================================================================
%=============================================================================================
\subsection{L-31N Subsurface Seepage Barrier}
%=============================================================================================
%=============================================================================================
\paragraph{Description}
The L-31N Seepage Barrier was built in two stages. The first 2 miles were completed in 2012, the last 3 miles (to make a total of 5 miles) were completed in April 2016.

\paragraph{Model Layout}

\begin{notes}
Planned model implementation is to change cell transmissivities in the top layers adjacent to the canal, and then use the sheetpile function for any layers below the lowest elevation of the canal.
\end{notes}

\clearpage


%=============================================================================================
%=============================================================================================
\subsection{Las Palmas Area}
%=============================================================================================
%=============================================================================================

%----------------------------------------------------------
\subsubsection{Las Palmas Levee}
%----------------------------------------------------------

\paragraph{Description}
Built sometime in 2006-2007.

\paragraph{Model Layout}
Not in model


\begin{notes}
Previously implemented as high-elevation MIKESHE cells.

Consider changing to Separated Overland Flow Areas in MIKESHE.
\end{notes}

%\clearpage

%----------------------------------------------------------
\subsubsection{Las Palmas Subsurface Seepage Barrier}
%----------------------------------------------------------

\paragraph{Description}
Construction begins in 2021.

\paragraph{Model Layout}
Not in model


\begin{notes}
For areas where the barrier is adjacent to a canal, implement as high-transmissivity cells with sheetpile underneath.
For areas where barrier is not adjacent to the canal, implement as sheetpile.
For areas where barrier extends above ground (such as when it is built into the levee, implement as SOLFA. (If it is in a levee, it should probably already be implemented as a SOLFA.)
\end{notes}

%\clearpage

%----------------------------------------------------------
\subsubsection{C-357 Canal}
%----------------------------------------------------------

\paragraph{Description}
Built sometime in 2006-2007.
Approximately 60 feet wide.

\paragraph{Model Layout}
Approximately 150 feet wide.  Bank heights approximately 1-2 feet above local LSE.

%\clearpage

%----------------------------------------------------------
\subsubsection{C-358 Canal}
%----------------------------------------------------------

\paragraph{Description}
Built sometime in 2013.
Approximately 25 feet wide.

\paragraph{Model Layout}
Approximately 150 feet wide. Bank heights approximately 4-5 feet above local LSE. Does not have MSHE links.

%\clearpage

%----------------------------------------------------------
\subsubsection{8.5 Square Mile Area Detention Cell}
%----------------------------------------------------------

\paragraph{Description}

Built sometime in 2006-2009.

The 8.5 Square Mile Area Detention Cell includes an inflow weir along the north end of the detention area (W-S359) which connects the detention area with the S-357 getaway. The weir is 400ft long and 9.5 NGVD29 high. Water is pumped from C-357 to getaway by the S-357 pump and reaches the detention area via the S-359 weir. Two out flow weirs (W-S360W and W-360E) are located along the southern end of the detention area. C-357 is a seepage collection canal. The S-357 pump operations control the water level within the canal and prevents overland flow.

To increase the flood protection of the 8.5 SMA, a canal (C-357) labeled S-357 in the model was constructed and connected to a pump station S-357, which discharges in the detention area NDA via a short flowway with weirs.


\paragraph{Model Layout}
The canal is implemented with a closed boundary on the north, a pump on the south, and recharge by groundwater infiltration. Overland flow is not active and does not directly collect surface water runoff.




\clearpage
%=============================================================================================
%=============================================================================================
\subsection{Northern Detention Area (NDA)}
%=============================================================================================
%=============================================================================================



\paragraph{Description}
The NDA is the 2018 configuration of the detention area between the 8.5 SMA Detention Cell and the SDA. From 2002-2017, a portion of this area was used, and was called S-332BN.

The S-332BN detention area had an emergency out flow weir along the eastern side.

\paragraph{Model Layout}

The configuration of the NDA is changed through different canal topographies and separated flow areas, as well as turning on or off certain structures.

The completed NDA is the active configuration in the operational model. It receives water from the 8.5 SMA detention area on the north and from the S-332BN pump on the southern end. The S-332BN overflow weir is not represented.



\clearpage
%=============================================================================================
%=============================================================================================
\subsection{Southern Detention Area (SDA)}
%=============================================================================================
%=============================================================================================

\paragraph{Description}

S-332BW: The north side of the S-332BW detention area includes a control structure to release discharge from the S-332BN detention area. The west side of the area has an emergency outflow weir. The east side includes 8 culverts and a 350ft long, 9.5ft NGVD29 high weir.

S-332 Connectors: In 2002 the S-332B partial connector and the S-332C partial connector cells were constructed. Each partial connector cell was connected to the respective western cell with a weir and set of several culverts. The middle portion of the connector cell was built in 2010, connecting the S-332B partial connector and the S-332C partial connector to form the full connector cell between BW and C.

Current construction includes only the . The USACOE plans to join the partial connectors and create one large southern detention area (SDAC). The model currently implements the entire SDAC detention area.



S-332C: The S-332C detention area receives flow from pump S-332C discharging from L-31N. An overflow weir along the southern east side of the detention area provides a discharge outlet.

The pumps from S-332BW, the westward facing components of the field station S-332B deliver water into the Southern Detention Area (SDA), which is made up of former farm lands (scraped down to remove the soil) and the natural, never farmed, areas.
In addition, the SDAC connector canal/flowway (constructed because some of the lands were not yet purchased at the time of construction) is also modeled, with a set of weirs connecting it to SDA. A fully open connector is modeled.
The entry and exit points of the connector contain both weirs and culverts in the field.

Details of the detention areas are shown in Fig.~\ref{fig:rjffig63}.

\begin{figure}[!h]
  \begin{center}
  \includegraphics[width=5in]{rjffig63.png}
  \caption[S-357, S-332BN and SDA Detention Areas Details.]{S-357, S-332BN and SDA Detention Areas Details. The re-alignment of the S-357 detention area with the grid (upper left) and the L-357 levee (as a topographic feature) with the C-357 canal (upper right). The S-332BN detention area (bottom left) and the emergency over  flow weir is shown with the location of the S-332BN pump. The SDA and SDAC detention areas with the S-332C pump are presented in the bottom right.}
  \label{fig:rjffig63}
  \end{center}
\end{figure}


\paragraph{Model Layout}




\clearpage
%=============================================================================================
%=============================================================================================
\subsection{L-31W Canal Area}
%=============================================================================================
%=============================================================================================

\paragraph{Description}
The L-31W canal is located in western Miami-Dade County along the eastern boundary of Everglades National Park (Figure~\ref{fig:L31W_Overview}).  The canal was originally constructed in 1968 to deliver water to Taylor Slough, but was later used to provide drainage to agricultural areas between the canal and the C-111.  Originally there were two structures in the canal. The S-174 delivered water from the L-31N canal into the L-31W.  The S-175 structure was operated as a drainage divide and allowed drainage from the basin into Everglades National Park south of SR-9336.  A pump station, S-332, was constructed in 1981 to pump water directly into the headwaters of Taylor Slough.  In 1992 the capacity of the pump station was increased by adding the S-332i.  After the S-332 structures were found to be ineffective at delivering water, a new pump station and detention basin was added just south of S-174.  The S-332D pump does not deliver water directly to the L-31W but rather to the Frogpond area to the east of the canal.

\begin{figure}[!h]
  \begin{center}
  \includegraphics[width=5in]{L31W_Overview.png}
  \caption[Map of L-31W Canal and surrounding area.]{Map of L-31W Canal and surrounding area.}
  \label{fig:L31W_Overview}
  \end{center}
\end{figure}

The L-31W canal was broken into 6 reaches based on the East-West or North-South orientation of the canal.  The first reach begins upstream of the S-174 structure and continues to the point where the canal turns to the south.  Reach 2 begins after the first bend to the south, and ends when the canal turns to the west, just south of C-328.  The third reach is the east-west section that begins south of C-328 and conveys water to the west.  Reach 4 is the north-south part of the canal that contains the levee removal zone on the east and west sides of the canal near Taylor Slough and the S-332/S-332I structures.  The east-west stretch connecting the S-332 reach of the canal to the reach that contains S-175 is Reach 5.  The bottom section of the canal, Reach 6, contains S-175 and conveys water south and under SR-9337.

A levee was constructed along the entire eastern side of the canal to protect the Frog Pond area from flooding associated with water deliveries to the Park.  There is not a levee along most of the western side of the canal; however several areas have banks that were built up slightly to provide vehicle access to structures or other features.  This includes the southern area of reach 4, all of reach 5, and small stretches of reach 6.

\paragraph{Model Layout}

A series of 49 cross sections were defined for the L-31W canal in the M3ENP model to define the L-31W topography circa 2002 for the IOP model run.   The data for the cross sections was obtained from two sources: as-built drawings done by Caribbean Land Surveyors Inc. in 1994 as part of the Hurricane Andrew Rehabilitation of Taylor Slough Basis and the M3ENP surface topography \citep{L31Wtopo}.   The as-builts were used to define the elevation profile of the canal and the levees.  The model surface topography is used to describe the banks where the canal does not have levees or in locations where the levee has been removed.  Cross sections for the plugs will be added at part of the L-31W topography circa 2017.

Cross sections were placed up and downstream of all structures located on the L-31W (S-174, S-175), and at the canal intersection of all structures that withdraw or input water to the L-31W but are not located in line with the canal (S-332DX1, SDA, S-328, Frog Pond Gap, S-332/S-332I, G-737).   Lowest point for the cross sections at the intersections was taken from the lowest point for the nearest cross section on the structure branch. Cross sections were also placed before and after every bend in the canal.  Finally, cross sections were placed 10m above and below each future location of a plug in the L-31W.

Cross sections for the as-builts were compared to 2 profiles from ADCP measurements taken in June of 2012 to confirm the general shape of the canal bottom and that the depth of the canal remains similar to earlier surveys (Figure~\ref{fig:ADCPcombo}).  The first ADCP cross section was taken on the L-31W near the location of the S-332D detention area headcell.  The tail water stage at S-332DX1 on the measurement date, 6/14/2012, was 6.20 feet NGVD29.  The bottom elevation of the canal was calculated to be -9.7 feet.  The canal has a flat bottom with sloping sides.  The as-builts show that the maximum depth of the L-31W in this area is variable and the maximum depth of the cross sections within 1000 feet of this location is between -9.93 and -12.13 feet NGVD29.

\begin{figure}[!h]
  \begin{center}
  \includegraphics[width=5in]{ADCPcombo.png}
  \caption[ADCP profiles of the L-31W canal. a. L31W canal near the bottom of S-332D headcell. b. L-31W canal south of SR9336 near the southern terminus.]{ADCP profiles of the L-31W canal. a. L-31W canal near the bottom of S-332D headcell. b. L-31W canal south of SR-9336 near the southern terminus.}
  \label{fig:ADCPcombo}
  \end{center}
\end{figure}

The second ADCP cross section was taken south of SR-9336, approximately half way between the road and the southern terminus of the canal.  S-175 tail water stage on the measurement date,  6/15/2012,   day was 2.94 ft.  The bottom elevation of the canal in this area was calculated to be  -9.66 feet NGVD29.  Plants on the side and bottom of the canal make it more difficult to get a good cross section.  The as-builts show that the maximum depth of the L-31W in this area is variable and the maximum depth of the cross sections within 1000 feet of this location is between -9.12 and -12.13 feet NGVD29.

Cross sections from the as-builts are 100 feet apart.  There is variability between cross sections, but the data from the cross sections was not analyzed to determine average depth or elevations.  A representative cross section was chosen based on length of the canal segment (Table~\ref{tab:asbuilt_crosssections}).  One cross section from the as-builts is used to define all the cross sections in short canal segments – such as section 1 and 3.  Long sections, such as 4 and 6, are defined using several cross sections that are selected from the area of the canal that cross section represents.

\begin{table}[!h]
\caption{As-built Cross Section used to create model cross sections in L-31W canal.}
\label{tab:asbuilt_crosssections}
\begin{tabular}{cc}
\hline
Reach & As-built Cross Sections                                         \\
\hline
1     & 536                                                             \\
2     & 575                                                             \\
3     & 651                                                             \\
4     & \begin{tabular}[c]{@{}l@{}}707\\ 756\\ 805\\ 845\end{tabular}   \\
5     & \begin{tabular}[c]{@{}l@{}}859\\ 908\end{tabular}               \\
6     & \begin{tabular}[c]{@{}l@{}}918\\ 972\\ 1017\\ 1062\end{tabular} \\
\hline
\end{tabular}
\end{table}

The elevation of areas on west side of the canal where there is either no, or a low levee, was defined using the elevation of the bordering M3ENP surface elevation grid cell.  Elevations on the as-builts vary, and are less than the surface elevation in some locations.  Several reaches have more than one grid cell that defines the elevation along the western side of the canal.  When more than one grid cell makes up the western boundary, the marsh is prescribed the value of the neighboring grid cell.

Portions of the levee have been removed at the headwaters of Taylor Slough.  The levee removal on the east side is approximately 2,000 feet (Figure~\ref{fig:LeveeRemovalZone}).  The levee removal on the west side is approximately 1,000 feet.  There is no source of reliable elevations for this area since the levee was removed after 1994 when the as-builts were drawn, and the grid cells along the boundaries of the canal incorporate a large area including the canal and levees.  The surface topography from the neighboring cell was used to define the east and west elevation of the levee removal zone.

\begin{figure}[!h]
  \begin{center}
  \includegraphics[width=5in]{leveeremovalzone.png}
  \caption[Cross sections at the Taylor Slough Levee removal area on the L-31W canal.]{Cross sections at the Taylor Slough Levee removal area on the L-31W canal.}
  \label{fig:LeveeRemovalZone}
  \end{center}
\end{figure}

\paragraph{Operational Model Layout}

The SFWMD has made several alternations to the L-31W canal and levee as part of the C-111 South Dade project.  Under Contract 9, 9 new plugs have been added to the L-31W canal.  With the exception of the plug in Reach 5, the plugs were constructed to match the adjacent marsh grade \citep{USACE2016C111SD}.  Two plugs had been added to Reach 1 in an earlier project, but the plug near the S-174 structure was expanded in 2017 ((Figure~\ref{fig:L31Wcanalcrosssections}).  A 940 ft plug was added on the southern end of Reach 2.  No plugs have been added to Reach 3.  Three plugs have been added in Reach 4, the farthest north is 1500 feet long, a 1000 feet long plug to the south, and a third plug that is 1,000 feet long located near the midpoint of Reach 4.  The plug placed on Reach 5 is 5,000 feet long and stretches most of the distance along the reach.   Instead of being at marsh level like the other plugs, the Reach 5 plug is located 2-3 feet below marsh level to allow conveyance from G-737 to the west into Taylor Slough.  Three new plugs have been installed in Reach 6: A 700 foot plug is located just south of S-175, a second plug (1,000 ft.) is located north of SR-9336, and a final 1000 ft. plug located south of SR-9336.

\begin{figure}[!h]
  \begin{center}
  \includegraphics[width=5in]{L31Wcanalcrosssections.png}
  \caption[Cross sections at the Taylor Slough Levee removal area on the L-31W canal.]{Cross sections at the Taylor Slough Levee removal area on the L-31W canal.}
  \label{fig:L31Wcanalcrosssections}
  \end{center}
\end{figure}


The plug cross sections were created using the elevation of the adjacent marsh from the M3ENP surface topography.  The elevation of the adjacent marsh is set, and the elevations of the points on the cross sections are reduced by 0.1 foot for every point along the cross section to the center of the canal ((Figure~\ref{fig:L31Wplugcrosssection}).  Plug elevations then increase by 0.1 foot until the cross section intercepts the adjacent levee or marsh surface.  Cross sections were placed in the canal at the beginning and ending of each plug.  Previously existing canal cross sections that are located within the footprint of a plug have been altered to reflect the plug topography.

\begin{figure}[!h]
  \begin{center}
  \includegraphics[width=5in]{L31Wplugcrosssection.png}
  \caption[Cross sections at the Taylor Slough Levee removal area on the L-31W canal.]{Cross sections at the Taylor Slough Levee removal area on the L-31W canal.}
  \label{fig:L31Wplugcrosssection}
  \end{center}
\end{figure}


The eastern levee gap near Taylor Slough (Reach 4) was also altered in 2017.  Previously the eastern gap was approximately 2,000 feet long.  As a result of modifications, this gap has been reduced to 500 feet and a weir was constructed +2 feet above ground level (GSE=4.0 feet NGVD29) (USACE, 2016).  The degraded levee segment has been rebuilt along the remaining 1,600 feet of the gap.  Existing cross sections have been modified to reflect the new elevation of the weir (6 foot NGVD29) or presence of the rebuilt eastern levee.  Several new cross sections were added to reflect the new location of the southern end of the weir.



\begin{notes}
\paragraph{Notes}
Need to sort out the first reach of L-31W where S-174 was replaced with a plug, and where the SDA canal intersects with L-31W.
\end{notes}

\clearpage

%=============================================================================================
%=============================================================================================
\subsection{S-332D Detention Area}
%=============================================================================================
%=============================================================================================
\paragraph{Description}

The former agricultural area between C-111 and L-31W, in large part, has been converted to a series of detention areas with associated infrastructure (Fig.~\ref{fig:rjffig60}). Adjacent and to the north of S-176 are the S-332D pumps designed to pump westward into a high-head cell with an over flow weir on the west side allowing flow to enter the western two sections of the Frogpond. The high-head cell stacks water to an elevation of 8 feet before over flow, thereby (because it is a highly permeable karstic geology underneath) resulting in large seepage into the C-111 canal. It has been estimated that in excess of 50\% of the pumpage returns to this canal as seepage.
On the western end seepage also entered L-31W directly downstream of corner plugs in the canal and also entered into the remnants of L-31W on the north side where, during high water levels, return flow goes directly across abandoned structure S-174. The S-174 structure was removed sometime in 2017 or 2018.
It is not clear what possible reason there is for a high-head cell in this environment.
The weir crest elevation was originally at 8.2 feet for a length of 1900 feet, which gave it a design of 500 cfs at 8.45 feet. A portion of this weir has since been lowered.

Flows that cross the weir and discharge into a flowway encounter another weir downstream set to over flow at 6.4 feet elevation.
Little flow enters the two western sections of the Frogpond under normal conditions.
Opposite the old pump station S-332 the embankment has been lowered to provide a direct exchange with the western sections allowing water to  flow east or west.
Lowering of the embankment on the west side, adjacent to S-332 facilitates  flow into Taylor Slough.


\begin{figure}[!h]
  \begin{center}
  \includegraphics[width=5.0in]{rjffig60.png}
  \caption[Frogpond canal/detention basin complex.]{Frogpond canal/detention basin complex. L-31W, S-332D and Frogpond detention areas. The Frogpond is a low-lying former agricultural area bounded by C-111 and L-31W Detention areas associated with the S-332D pump and high head cell allow flows to enter the western sections of the Frogpond.

  Three detention areas are downstream of S200 (FPDA-1, FPDA-2 and FPDA-3). The canal from the Pump S-200 is lined to the western corner in order to minimize seepage back into C-111 canal. The purpose of the detention areas are to maintain a higher hydraulic head in the area to minimize eastern groundwater flows from ENP to C-111, thereby encouraging flows into Taylor Slough.}
  \label{fig:rjffig60}
  \end{center}
\end{figure}


\paragraph{Calibration Model Layout}

The bottom map in Fig.~\ref{fig:rjffig61} is the .nwk11 representation of the southern part of the Frogpond system.
The flowway Fp delivers water from the high-head cell to the western section of the Frogpond, where the canal reach intersects with the MSHE cells for both surface water flow and groundwater flow.

Fig.~\ref{fig:rjffig61} are two detailed representations of the .nwk11 M11 canal layout file.
The top map shows the configuration of the S-332D high-head cell and adjacent canals.
The southern terminus of the Southern Detention Area (SDA), the start of the Frogpond (FP) flowway and the northern reach of L-31W are shown.
S-332D pumps from L-31N canal, north of S-176, into the high-head cell and S-176 structure controls southward  flow into C-111.
The pump station is capable of delivering 575 cfs using four diesel pumps at 125 cfs each and on electric pump at 75 cfs.
The lowered embankment on the west side of L-31W, adjacent to S-332, is modeled with a short canal to the southwest labeled S-332 in the model (Fig.~\ref{fig:rjffig61}).

The S-332D high head cell receives discharge from the S-332D pump. The high head cell is implemented as an additional canal and the area of the cell is defined by the canal cross sections. There is a weir along the south side of the cell which is 1900ft long and 8.1ft NGVD29 high. The weir connects the high head cell with S-332D cell 1.

The S-332D Cell 1 detention area has an earthen berm along the south side with a length of 2100ft and a height of 6.5ft NGVD29. The berm is implemented as a weir and connects Cell 1 with Cell 2. Cell 2 has a 1900ft long, 6ft NGVD29 high, weir along the south side. The weir connects Cell 2 with S-332D Cell 3.

S-332D Cell 3, also called Frog Pond or Flow Way, is the southern most detention area. Flow is received across the weir joining the S-332D Cell 2 and Cell 3. The canal implemented in the model to represent  flow through the detention areas ends in the center of Cell 3.

\begin{figure}[!h]
  \begin{center}
  \includegraphics[width=2.5in]{rjffig61.png}
  \caption[Details of the Frogpond canal complex.]{Details of the Frogpond canal complex.

  Top: The northern portion of the Frogpond includes the high-head cell (S-332D) represented as a canal with cross section from field surveys. The L-31W canal was eliminated from the intersection with L-31N (at S-174) to facilitate the M11/MSHE interaction of the high-head cell in the 4000m model, since both canals fall on the same cell interface.

  Bottom: Detail of the area from the .nwk11 file, showing the central Frogpond area. Both the deep canal L-31W and the shallow flowway FP convey water from north. FPgap is a short canal representing the elimination of the L-31W western berm allwing exchange between canal and flowway. S-332 takes L-31W water and distributes it to MSHE cells into Taylor Slough. Each of the Frog Pond Detention Area cells have weirs from the inlet canal, FPDA-INLET, connected to pump station S-199.}
  \label{fig:rjffig61}
  \end{center}
\end{figure}


\paragraph{Operational Model Layout}
In the operational model, the groundwater link for the High Head Cell has been shortened to account for the ``lining'' of the getaway from the S-332D pump station in 2017-2018. Also, the S-332D High Head Cell weir S-327 is changed to it's new configuration to account for the area where a portion of this weir was degraded.

The tail end of Cell three should be modified to reflect the new weir and flow barrier separating Cell 3 from the L-31W.

\clearpage


%=============================================================================================
%=============================================================================================
\subsection{Frog Pond Detention Area}
%=============================================================================================
%=============================================================================================

\paragraph{Description}
Built in 2013.

\paragraph{Model Layout}



\clearpage
%=============================================================================================
%=============================================================================================
\subsection{Aerojet and Aerojet-EXT Area}
%=============================================================================================
%=============================================================================================

\paragraph{Description}
Aerojet Ext and S-199 were built around 2013.
A lined canal reach starts from C-111 just north of S-177 (Fig.~\ref{fig:rjffig65}).
The reach receives water from C-111 via the S-199 pump station.
A weir ends the lined reach and the canal flows into the next reach labeled Aerojet.
This canal has weirs at chainages 5600, 8700 (Aerojet-rock1), 14000 (Aerojet-rock2) and 25700.

\paragraph{Model Layout}



\begin{figure}[!h]
  \begin{center}
  \includegraphics[width=5in]{rjffig65.png}
  \caption[Aerojet and Aerojet-EXT.]{Aerojet and Aerojet-EXT. A profile is shown in the bottom insert detailing the weirs in the canal as elevated cross sections. The .nwk11 representation is shown in the upper right plan view.}
  \label{fig:rjffig65}
  \end{center}
\end{figure}



\clearpage

\cleardoublepage

